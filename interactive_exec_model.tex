DERP is designed as a language that is easy to use for anyone that wants to get a personalized news feed. To accomplish the goals of this language we needed the language itself to handle a lot of the nuances of the data we will be gathering, as well as allowing the programmer to focus on creating the content that goes into the queries. The programmer needs to worry less on memory, but more on the actual logic of the task at hand. The execution model of DERP will take all of this into account. 



\subsection{Interactive}
The execution of statements within the DERP language will be done with a Read-Eval-Print-Loop (REPL). A statement in our language will be a single line where the delimiter between statements is a newline characters. This REPL will go line by line from typed in text that can either be done manually or through redirected input. For the puposes of creating a prototype this will be done in a text interface similar to the REPL of python and bash. As time permits we will implement this using voice as the input. This will be similar to the REPL, but we will refer to it as the Hear-Eval-Speak-Loop (HESL).  
