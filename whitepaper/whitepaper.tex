\documentclass{article}
\usepackage[margin=1in]{geometry}
\usepackage[utf8]{inputenc}
\usepackage{comment}
\usepackage{listings}
\usepackage{titlesec}       % Lines under Sections
\titleformat{\section}
  {\normalfont\Large\bfseries}{\thesection}{1em}{}[{\titlerule[0.8pt]}]

\usepackage{etoolbox}       % Because titlesec removes section numbering -_-
\makeatletter
\patchcmd{\ttlh@hang}{\parindent\z@}{\parindent\z@\leavevmode}{}{}
\patchcmd{\ttlh@hang}{\noindent}{}{}{}
\makeatother

%\renewcommand\maketitle{}

% Give Table of Contents Hyperlinks
\usepackage{hyperref}
\hypersetup{
    colorlinks,
    citecolor=black,
    filecolor=black,
    linkcolor=black,
    urlcolor=blue
}
\pagenumbering{gobble}
% \pagenumbering{roman} % set the numbering style to lowercase letter\autoref*{#1} 
\newcommand*{\fullref}[1]{\hyperref[{#1}]{section \ref*{#1} \nameref*{#1} }}
\title{\textbf{Dictation Evaluation Reddit Parser}}

\author{
Garcia, Benjamin \and
Lembke, Logan \and 
MacMillan, Kyle  \and 
Smith, Christopher \and 
Stelter, Andrew 
}
\date{September 21, 2018}


%\setcounter{secnumdepth}{4}

\titleformat{\paragraph}
{\normalfont\normalsize\bfseries}{\theparagraph}{1em}{}
\titlespacing*{\paragraph}
{0pt}{3.25ex plus 1ex minus .2ex}{1.5ex plus .2ex}

\begin{document}


\addcontentsline{toc}{section}{Title}
\maketitle %%TODO: maketitle doesn't seem to be working

\newpage
\tableofcontents
% \addcontentsline{toc}{section}{Table of Contents}

\pagenumbering{roman}   % Set TOC page numbering to lowercase roman numerals



%%%%%%%%%%%%%%%%%%%%%%%%%%%% INTRO SECTION %%%%%%%%%%%%%%%%%%%%%%%%%%%%
\newpage
\hypersetup{
    colorlinks,
    citecolor=blue,
    filecolor=black,
    linkcolor=blue,
    urlcolor=blue
}
\pagenumbering{arabic}  % Set content page numbering to arabic numerals

\section{\textbf{Introduction to DERP}}

\setcounter{page}{1} % Set the page counter to 3
\href{https://gitlab.mcs.sdsmt.edu/7184015/DERP}{DERP}, the \textbf{D}ictation \textbf{E}valuation \textbf{R}eddit \textbf{P}arser, is a \href{https://en.wikipedia.org/wiki/Domain-specific_language}{DSL}(Domain Specific Language) for finding news information from multiple web sources simultaneously. The system was designed as part of a group project for South Dakota School of Mines \& Technology's Compilers course.

\subsection{What is DERP: Human Queries For a Digital World}\label{sec:whatisDERP}
Time is a resource that people regularly find themselves lacking, and many would agree it would be convenient to have an easy way to consume online news while doing daily tasks -- preparing and eating breakfast, taking a shower, getting dressed, or driving to work. We should be able to say to a phone, ``What's going on in the world today?'' and get a response containing relevant information that we care about; not just ``I'm sorry, I don't understand, `What's going on in the world today' ''. If we want news on a specific topic, such as Tesla, getting that information should be as easy as saying, ``What's in the news about Tesla?''. 

The formats used to present news are largely unchanged in recent years; sure, the internet provides a way for news to be shared, but we still have to hunt down the stories we want, whether we get them through a radio or television broadcast, an individual online or print article, or a newspaper. These formats are often quite rigid, giving us little control over what news we get, or so full of information that doesn't concern us that it can be difficult to know where to begin. One common solution used as a middle ground to these problems is aggregation sites such as \href{https://www.reddit.com}{Reddit}, \href{https://steemit.com/}{steemit}, \href{https://band.us/home}{band}, and \href{https://voat.co/}{voat} to name a \href{https://www.reddit.com/r/RedditAlternatives/comments/8585ox/list_of_active_reddit_alternatives/}{few}. While these sites in this category go a long way toward solving the problems presented, they often fall into the same traps: Lacking information users want or presenting so much information that it's difficult to wade through it all.

\subsection{DERP Origins}
In the fall of 2018 we were tasked with making a DSL for a class in compiler theory. A number of applications discussed by our team follow:
\begin{itemize}
    \item{Describe Constraints  to Generate something}
    \begin{itemize}
        \item{Floor Plans}
        \item{Workout Assistant}
        \item{Computer Part Picker}
    \end{itemize}
    \item{Perform a Task}
    \begin{itemize}
        \item{Robot "AI" Proposal by Dr. Hinker}
        \item{Simple Image Processing}
    \end{itemize}
    \item{Configure a Task}
    \begin{itemize}
        \item{News Reader}
        \item{SQL Helper}        
    \end{itemize}
\end{itemize}

We decided against a Constrain/Generate model because the ideas we came up with all required labeled data which we did not have access to, and we were not sure that we would be able to find or make sufficient data in a timely manner.

Creating a language to help perform a task seemed like a good idea at first, but we realized being reliant on some external item (a robot, in the case of Dr. Hinker's proposal) could hinder progress unnecessarily. We were also wary of working with a new, possibly unfinished, project and didn't know if we'd be walking into a well-documented, fully-defined product or a hack. Going down the route of something purely software, as we would have with our image processing concept, was a more viable option for this category of project, but we ultimately decided against it.

\begin{comment}
Do we want to talk about why we didn't go with the image processing project?
\end{comment}

Lastly, we discussed configuration-style tasks. The two proposals from team members were for a News Reader and for a SQL Helper. The core concept of the News Reader was that a user should be able to create custom queries that will obtain only the news that they are interested in, regardless of where the information was originally from. The idea behind the SQL Helper was to make SQL more accessible for the general public to use in everyday business situations; for example, a secretary with no prior SQL knowledge should be able to use it to get information from the company database to generate a report. We deemed the headache of building such a system while also making it robust enough to translate human queries into SQL to be too much work for the amount of time given for the project; so, we settled on the News Reader.

Every project needs a name; having selected our project, this was the most important task left to us. We settled on \textbf{D}ictation \textbf{E}valuation \textbf{R}eddit \textbf{P}arser because, if we have enough time, we'd like it to take spoken words as input to evaluate. The project will focus on interfacing with Reddit for a minimum viable product; however, the language is designed with other sources in mind.


\subsection{Why DERP}
DERP allows a developer to create an interface between a person and information feed sites such as \href{https://www.reddit.com}{Reddit}, \href{https://steemit.com/}{steemit}, \href{https://band.us/home}{BAND}, and \href{https://voat.co/}{Voat}, as described in the \hyperref[sec:whatisDERP]{What is DERP} section. This interface allows for a programmer to develop an application that cleanly links an end-user to the feed site(s) of their choice, without the user even being aware that they are using DERP. While developers who want to access new news sites will still need to know how to get information from the source site of their choosing, DERP is designed to do the heavy lifting of finding and sorting relevant data, allowing less technically-minded users to build their own information filters from existing sources. It is a free solution which can provide a consistent starting point for developers so that they do not need to worry about the details of determining what should be searched and how results should be organized.


\newpage
\section{DERP Design Goals}
\subsection{Consolidated Information}
The DERP project provides an intuitive way to multiplex groups of news sources into a personalized stream of data. Using DERP, a user can specify as many different sources as they want to (provided language plugins are available), and then all of those sources are accessed through the same set of language keywords, regardless of if the source is a subreddit, an arbitrary news website, or even just a file on the user's device. Furthermore, DERP facilitates naming groups of sources and queries to create criteria and new sources composed of other sources that can be used together, allowing users to further personalize what they get out of DERP.

Online news sources regularly expose the same categories of information - date, author, title, etc. DERP acts as an interface for all different article types, allowing users to work only with this high-level information about the articles they are finding. Hiding the details of what exactly is required to find an article that matches a specific criterion allows users to focus more on what they want to read, and less on how they obtain that information.

\subsection{Easy-to-Use Natural English Interface}
The second goal of DERP is to design a language that would not feel awkward if spoken. All of the DERP keywords and syntax follow simple but natural English speech patterns. This allows for easy adaptation of DERP into speech-recognition tools such as Google Assistant and Amazon Alexa. Using the natural form of the language, users with one of these devices could speak their program to the interpreter and receive results immediately.

In a similar vein, DERP provides output that also feels natural.
DERP has a set of phrases available to it for reporting errors or results from user queries. The main output from DERP, articles the user has requested, are outputted as full text so that, should devices reading them use a text-to-speech system, they will sound as natural as possible.

\subsection{Search More, Faster}
By merging news sources into a personalized feed, DERP allows users to find the information they want with fewer operations. Rather than check each of their favorite subreddits, news sites, and RSS feeds, a user can simply load those sources into DERP and make a general query that applies all of them at once. DERP will do the hard work of finding things the user might be interested in, which means the user will be able to spend more of their time actually consuming the information provided and deciding on new topics to get information about.

Because it is an extensible language, DERP's users are limited only by the language plugins they use. While some keywords are understood specially by the language, such as those pertaining to dates and times, any language plugin can provide additional fields and keywords, making the language flexible and powerful while keeping complexity out of it's core.


%%%%%%%%%%%%%%%%%%%%%%%%%%%% EXEC MODEL SECTION %%%%%%%%%%%%%%%%%%%%%%%%%%%%
\newpage
\section{\textbf{The DERP Execution Model}}
To accomplish the goals of DERP, the language itself will be required to handle a lot of the nuances of the data it is used to collect in addition to allowing users to focus on creating the content that goes into the queries. Users should need to worry less about memory and more about the actual logic of the task at hand. The execution model of DERP will take all of these points into account. 



\subsection{Interactive}
The execution of statements within the DERP language will be done through a Read-Eval-Print-Loop (REPL). This REPL will evaluate language statements sequentially from its  input stream, which can be any valid source of text. For the initial prototype, the DERP REPL will provide a text interface similar to the interactive environments of Python and Bash. If time permits, we will implement a second interface layer to accept speech as the input - Such an interpreter would be similar to the REPL, but we will refer to it as the Hear-Eval-Speak-Loop (HESL).  

\subsection{Hierachical Data (website $\rightarrow$ boards (subreddits) $\rightarrow$ posts}
DERP is a declaritive language, so the programmer is focused in the logic of the statements to perform queries, otherwise known as selections. This means that the heavy lifting is done by the interpretor that is running the REPL. The interpretor will allow a programmer (user) to use sources such as reddit ( if time allows we will add more sources such as cnn, fox, nbc, etc ). The interpretor itself will make sure that a statement is valid and conforms to the standars of the current mode and modules loaded for news sources. It then will evaluate the statement and perform a search of the loaded news sources with the criteria specified. Next it will initally pull only enough information for the the user to identify the article's defining information such as its title, date published, and source it is from. The user than can specify follow up statements that may perform summaries, new queries, give details about each article, and so on. \\


\subsubsection{Modes}
The interpretor will have 3 modes that depend on the keywords used in the statement. These modes are the main mode, selection mode, and criteria mode.

\paragraph{Main Mode}

The DERP interpretor will start in the main mode, also known as the top level mode. This mode allows the user to clear and read statements, as well as enter other modes. The other modes that can be entered are the selection and criteria creation modes. In the main mode the user would need to specify create new criteria or create new selection to enter these modes. To exit the interpretor the user would specify exit in this mode in order to quit out of the interpretor.

\paragraph{Selection Mode}

The selection mode would be able to add modules and add and remove queries in the statements, and also be able to save and read the query so far. To exit the mode, the user would specify stop, a keyword, to quit composing selections and enter the main mode again. Examples of using this mode are found in \autoref{subsub:Composable}. 

\paragraph{Criteria Mode}

The criteria mode allows for the user to create logical statements that use the add, remove, and save statements. To exit the mode, the user would specify stop, a keyword, to quit composing logical statements and enter the main mode again. Details can be found in \autoref{subsub:Composable}.


\subsubsection{Modules}

The interpretor will also allow for modules to be loaded during run time. These modules allow for the user to specify news sources with the news module and searchable criterial with the knowledge module. Examples of module use can be found in \autoref{sssec:Extensible}

\paragraph{News Module }

The news source is going to a be a modular component of the interpretor that can be defined by users. This will be a module in python that will define how the interpretor will interface with the source, what are the keywords associated with it, filters and search criteria will also be defined. For example reddit has a Python api that can be interfaced with and will have kewords that get defined such as subreddit and posts. It will also define what is considered a criteria, what a filter can exclude and so on. For simplicity, Reddit will be implemented initally and others added if time permits. When this becomes a HESL, the modules will be built into the interpreter and periodic updates will happen to allow for more sources of news to be suppported.\\


\paragraph{Knowledge Module}
The knowledge modules are used when the user wants to search through articles that were previously fetched using selection statements with news modules. This module would define how articles a are searched through and filtered locally. It would get the contents of the article at this point and filter tham further by the searchable criteria that the user stated.

\subsection{Language Constraints}
There are a few constraints on the interpreter implementation for the DERP-Query Language. These are mainly based around the keywords defined in the \autoref{sub:DerpOperators}.\\
\\The following must be defined for the main selection mode:

\begin{itemize}
    \item create 
    \item clear
    \item read
    \item exit 
\end{itemize}

\noindent The following must be defined for the selection and criteria modes:

\begin{itemize}
    \item add
    \item remove
    \item clear
    \item recall
    \item save as
    \item read
    \item stop
\end{itemize}

\noindent Also all modules loaded must must not define their own keywords that are the same as interpreter keywords.


%%%%%%%%%%%%%%%%%%%%%%%%%%%% DQL SECTION %%%%%%%%%%%%%%%%%%%%%%%%%%%%
\newpage
\section{\textbf{The DERP Query Language}}
The DERP Query Language (DERP-QL) makes it easy to write complex queries for a variety of Internet sources. In DERP, queries are represented by \textit{selections}. Selections find postings from a particular set of sources that match a corresponding set of criteria. The following is an example of a DERP-QL selection.

% Quotes in listings looks terrible
\begin{lstlisting}
Add posts from subreddit "nba" or subreddit "basketball" on "Lakers"
Remove posts with under 1000 upvotes
Remove posts on "Lebron James"
\end{lstlisting}

This selection does exactly what it says. It finds posts from \texttt{reddit.com/r/nba} or \\ \texttt{reddit.com/r/basketball} that don't involve Lebron James with 1000 upvotes or more.

\subsection{Main Features}
DERP's main features revolve around being composable, interactive, and extensible. 
\subsubsection{Composable}
\label{subsub:Composable}
DERP-QL is composable on several levels. First of all, selections are composed of several \textit{selection statements}.
Selections always start with an \texttt{Add...From} statement. \texttt{Add...From} statements supply at least one source to pull postings from. Additionally, they may restrict the postings to a set of criteria. While selections always start with an \texttt{Add} statement, the selection can be further altered by any number of \texttt{Add} or \texttt{Remove} statements. 

Once a selection has been built, it can be saved under a given name. Any saved selection can then be used as a source for another selection. For example, the above example could be split into multiple selections.

\begin{lstlisting}
Create a new selection
    Add posts from subreddit "nba" or subreddit "basketball" on "Lakers"
    Save as "Lakers news"
Stop
Create a new selection
    Add posts from "Lakers news"
    Remove posts with under 1000 upvotes
    Remove posts on "Lebron James"
    Save as "Lakers news without Lebron"
Stop
\end{lstlisting}

Composing selections makes it easy to assign a name to a number of sources. For example, the first selection in the example immediately above could be broken into two selections.

\begin{lstlisting}
Create a new selection
    Add posts from subreddit "nba" or subreddit "basketball" 
    Save as "basketball subreddits"
Stop
Create a new selection
    Add posts from "basketball subreddits" on "Lakers"
    Save as "Lakers news"
Stop
Create a new selection
    Add posts from "Lakers news"
    Remove posts with under 1000 upvotes
    Remove posts on "Lebron James"
    Save as "Lakers news without Lebron"
Stop
\end{lstlisting}

While reusing selections is handy, it is even more useful to reuse the same filter criteria across multiple sources.
While selections must pull from at least one source with an \texttt{Add...From} statement, \textit{criteria}\label{create:criteria} are purely logical constructs. \textit{Criteria} provide a means to compose search criteria without pulling posts from any individual source. The selection above can be rewritten in terms of criteria.

\begin{lstlisting}
Create new criteria
    Add posts on "Lakers"
    Remove posts with under 1000 upvotes
    Remove posts on "Lebron James"
    Save as "Lakers without Lebron"
Stop
Create new selection
    Add posts from subbredit "nba" or subreddit "basketball" \ 
        matching "Lakers without Lebron"
    Save as "Lakers news without Lebron"
Stop
\end{lstlisting}
Note: ``\textbackslash'' is not a part of DERP-QL. However, ``\textbackslash'' here means the statement continues on the next line.

\subsubsection{Interactive}
DERP-QL is meant to be used in an interactive manner either through a REPL (Read Evaluate Print Loop) or a HESL (Hear Evaluate Speak Loop). DERP-QL specifies keywords and semantics that a DERP-QL REPL/HESL must support. Of particular note are the \texttt{Read} and \texttt{Recall} keywords. \texttt{Read} signals the interpreter to execute a selection. For example, \texttt{Read "Lakers news without Lebron"} pauses the interpretation of the containing DERP-QL program, and begins the interpreter dependent handling of the stored selection ``Lakers news without Lebron''. The \texttt{Read} command may also be used while creating a new selection to check the output of the current selection being built.

\texttt{Recall}, on the other hand, makes programming easier for voice users. \texttt{Recall} causes the interpreter to list out the statements which make up a given selection. This command is especially helpful in the selection and criteria creation environments when programming by voice. 

In order to promote human interaction, several words are parsed as filler including lone articles \texttt{a, an, the} and the word \texttt{with}.

\subsubsection{Extensible}\label{sssec:Extensible}
Source modules may define their own grammars which are concatenated into the overall language. For example, the reddit source module defines the \texttt{subreddit} keyword. The subreddit keyword takes the next string and uses it to select a portion of reddit to use as a source.


\subsection{Primitives}
There are N primitives in the language
\begin{enumerate}
\item Date - field type used for queries based on date fields
\item Number - field type used for queries on numeric fields
\item Boolean - field type used for queries on true/false fields
\item String - field type used for queries on text fields
\end{enumerate}
\subsection{Higher Order Types}
\begin{enumerate}
\item Selection - Composed of one or more selection statements. Contains at least one \texttt{Add} statement.
\item Criteria - Composed of one or more criteria statements (boolean expressions). Represents a predicate for whittling down the results from a selection. 
\end{enumerate}
\subsection{Memory Management}
Memory management is automatic, and handled by the Python runtime. Selections, and criteria are persisted to files, and loaded into their representative objects at runtime. When selections or criteria are composed, they are composed by value rather than reference. This prevents dependency issues when criteria or selections are removed from the system.
\subsection{Criteria Matching Operators}
\subsubsection{String Operators}
\label{sub:DerpOperators}
\begin{itemize}
\item with the exact - String comparison equality operator. Includes results where the specified field is an exact match for the provided string.
\item like - String comparison fuzzy equality operator. Includes results where the specified field is an approximate match for the provided string.
\item in the - Sub-String comparison equality operator. Includes results where a sub-string of the specified field is an exact match for the provided string.
\end{itemize}
\subsubsection{Datetime Operators}
\begin{itemize}
\item Date on - Date comparison equality operator. Includes results where the specified field is exactly the same date as the provided date.
\item date after - Date comparison greater-than operator. Includes results where the specified field is strictly after the provided date.
\item date before - Date comparison less-than operator. Includes results where the specified field is strictly before the provided date.
\end{itemize}
\subsubsection{Boolean Operators}
\begin{itemize}
\item which are $|$ are - Boolean comparison equality operator. Includes results where the specified field contains the boolean value of `true'.
\item which are not $|$ are not - Boolean comparison non-equality operator. Includes results where the specified field contains the boolean value of `false'.
\end{itemize}
\subsubsection{Numeric Operators}
\begin{itemize}
\item with exactly $|$ exactly - Number comparison equality operator. Includes results where the specified field contains the same numeric value as the provided numeric value.
\item with over $|$ over - Number comparison greater-than operator. Includes results where the specified field contains a value strictly greater than the provided numeric value.
\item with under $|$ under - Number comparison less-than operator. Includes results where the specified field contains a value strictly less than the provided numeric value.
\item with roughly $|$ roughly - Number comparison epsilon equality operator. Includes results where the specified field contains a value within an epsilon value greater or less than the provided numeric value.
\end{itemize}
\subsubsection{Misc Operators}
\begin{itemize}
\item matching - Criteria composition operator. The criteria corresponding to the provided name will be textually included into the current criteria or selection.
\item on - Substring search on "topical" fields as defined by a source module.
\item and $|$ or - Combine the results from the statements on either side of the operator.
\end{itemize}

\subsection{Selection Matching Operators}
\begin{itemize}
    \item from - Source selection operator/ Selection composition operator. Pulls posts from a registered source or a previously saved selection. 
    \item All criteria matching operators.
\end{itemize}

\subsection{Keywords}

\subsubsection{Main Mode}
\begin{itemize}
\item exit - Exits the program, and may only be used in the mode selection mode.
\item clear - Deletes a specified selection or criteria.
\item load/ unload - Loads a news source module or a knowledge module.
\item recall - Read out the selection statements or criteria statements which make up a specified selection or criterion.
\item read - Execute the specified selection and present the results.
\item create - Creates a new selection or criterion. Enters the respective builder mode.
\end{itemize}

\subsubsection{Selection and Criteria Builder Mode}
\begin{itemize}
\item stop - End mode keyword. Ends selection or criteria creation mode and clears the active state.
\item clear - Reset mode state keyword. Without leaving the current creation mode, clears the active state.
\item recall - Read out the selection statements or criteria statements which make up the object that is currently being built.
\item read - (selection builder only) Execute the selection currently being built and present the results.
\item save as - Store the current selection or criteria with a specified name. A saved selection or criteria may be used in the creation of other selections and criteria.
\item add - Select posts which match a set of criteria. \texttt{From} is disallowed in in add statements in the criteria builder mode.
\item remove - Remove posts which match a set of criteria.
\end{itemize}


\end{document}
