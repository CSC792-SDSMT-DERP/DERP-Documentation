\section{Language Modes}

The DERP language is defined in terms of different modes of operation.
Switching to a new mode is achieved through the use of certain expressions
which are specific to the current mode. When a mode is triggered, the
subset of the language that is valid may change. The next sections give
a brief description of each mode, as well as the statements used to switch between them.


\subsection{Main Mode}

As the name implies, this is the mode that a DERP interpreter should begin in. In this mode,
statements that change the loaded plugins, output the instructions for a saved query, read a
saved query, clear the currently saved result, or begin query/criteria creation are all valid.
See section [section \#] for more details about Main Mode statements.
%%%%%%REFRENCE A SECTION NUMBER%%%%%%%%%%%%

Use a \ecode{create\_expression} with the selection keyword to switch from Main Mode to Selection Mode
\begin{description}[labelindent=1cm,leftmargin=\onelen,labelwidth=1cm]
    \myitem{\textit{create\_expression}}{:} \texttt{create \textit{article}\textbf{?} new\textbf{?} selection}
\end{description}

Use a \ecode{create\_expression} with the criteria keyword to switch from Main Mode to Criteria Mode
\begin{description}[labelindent=1cm,leftmargin=\onelen,labelwidth=1cm]
    \myitem{\textit{create\_expression}}{:} \texttt{create \textit{article}\textbf{?} new\textbf{?} criteria}
\end{description}

To end your DERP program or close the DERP interpreter use a \ecode{stop\_expression} in Main Mode
\begin{description}[labelindent=1cm,leftmargin=\onelen,labelwidth=1cm]
    \myitem{\textit{stop\_expression}}{:} \texttt{stop \textbf{\textbar} exit}
\end{description}

\subsection{Selection Mode}

Selection mode is used to create a new selection statement which can be executed at a later time from Main Mode.
While in Selection Mode, statements which build the query, read the query statements, save the query, or exit
Selection Mode are all valid. See section [section \#] for more details about Selection Mode statements.
%%%%REFREENCE A SECTION NUMBER %%%%%%%%%%%%%%%%%%

To exit Selection Mode (and return to Main Mode), use a \ecode{stop\_expression}.

\subsection{Criteria Mode}
Criteria mode is very similar to selection mode; it is used to create criteria rather than selections.
The difference between a criterion and a selection is that a criterion cannot be specific to any source;
it is a filter that can be applied to a selection, but it is not a valid selection on its own.
See section [section \#] for more details about Criteria Mode statements.
%%%%%REFRENCE A SECTION NUMBER%%%%%%%%

To exit Criteria Mode (and return to Main Mode), use a \ecode{stop\_expression}.
