\section{Data Types}

\subsection{Post}
Posts are the data type that DERP is built around. A Post is nothing more than some body 
of text and any number of fields containing information about the post. Fields of a post 
are key-value pairs mapping some name for the field to its value.

While defining the interface of a post is the task of a DERP interpreter writer, some of 
the functionalities it will likely be required to have are described here.

\subsubsection{Retrieving Text Body}
Posts should provide some method by which the DERP interpreter can acquire the entire 
text in the body of the post. It is not required that this information be stored directly 
in the post object itself, however -- Retrieving this data can be delayed until it is 
requested by the DERP interpreter.

\subsection{Source Modules}
The first data type that someone writing DERP code will encounter is the Source Module. 
Source Modules are not created directly by the programmer, but are loaded and unloaded to 
modify the language. A source consists of, at minimum, new syntax for defining how to 
reference the source, and a public code interface that the DERP interpreter can use to 
get results from online.

Source modules are allowed to define new syntax rules for the following DERP nonterminals:
\begin{itemize}
\item source
\item field
\item sortkey
\end{itemize}
When input is found to match the rule a module provided for source, it indicates that the 
source module will be able to convert that input to a set of posts.

Source modules are allowed to populate an arbitrary set of key-value fields in the posts 
they create. Defining new syntax for \emph{field} extends the DERP parser to be able to recognize 
the keys that the source module may populate in posts it retrieves.

By default, every field is also a sort key; however, it may be desirable to sort posts according 
to some ordering other than one of their fields. Defining new syntax for \emph{sortkey} extends the 
language to recognize these new sorting methods.

While defining the interface of a source is the task of a DERP interpreter writer, some of the 
functionalities it will likely be required to have are described here.

\subsubsection{Loading Grammar}
As stated above, source modules are allowed to define new syntax and use that syntax to overload parts 
of the DERP language; therefore, any interface to the source module will likely provide a way for the 
DERP interpreter to get those changes to the grammar when the module is loaded.

It is anticipated that the grammar definition provided by a module may involve extra definitions of 
the fields added. These extra definitions can include information such as the data types associated 
with those fields. Without such a definition, DERP interpreters will not be able to perform type-checking 
on \emph{qualifiers} before attempting to executing them.

\subsubsection{Performing Queries}
The most important part of a source module is its ability to take the input that was parsed according to 
source syntax and produce posts. Because source modules will likely deal with very large databases of 
information that posts can be retrieved from, this interface will likely need to provide functionalities 
to start retrieving posts from a specific point in the database and to retrieve a finite number of results 
from there.


\subsubsection{Sorting}
If the source module defines some non-standard \emph{sortkey} it will likely also need to provide some method by 
which the interpreter can sort things when that \emph{sortkey} is specified.

\subsubsection{General Matching}%%%%%%%%REFRENCE NUMBER%%%%%%%%%%%%%%%%%
One of the special qualifiers defined in [section #], the \emph{on_exp} qualifier, is used as a general-purpose 
match of posts against some text. If the interpreter does not implement some sort of generic matching 
routine, it will likely be the responsibility of source modules to determine if a post satisfies this 
generalized match.

\subsection{Criterion}
A criterion is a user-defined filter for specifying a subset of the Posts returned from executing a query. 
Criteria are created by entering Criteria Mode, making a number of statements to define the criterion, and 
then saving the criterion with a \emph{save_expression}.

\subsection{Selection}
A selection follows many of the same rules as a Criterion, with the important difference that it can 
reference a specific source using the syntax rules defined in some source module. Because Selections 
contain one or more sources to get Posts from, they can be executed with a read statement to return a 
set of posts.
