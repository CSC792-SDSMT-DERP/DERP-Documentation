\section{Syntax Notation}
Syntax definitions in this reference will be separated from surrounding text by an empty line,
indented, and written in a monospace font. The grammar definitions will follow Backus-Naur form,
occasionally using Regular Expression syntax for brevity. Additionally, definitions will use the following conventions:

\begin{itemize}
\item Terminals will be unformatted; keywords will be expressed as a sequence of terminal characters.
\item Nonterminals will be \textit{italicized}
\item Characters which could be considered either terminals or BNF/Regular Expression syntax will be \textbf{bold} when they should be interpreted as part of BNF or Regular Expression syntax.
\item All terminals shown are lower-case text; however, DERP parsers should match them case-insensitive.
\item Production definitions follow the format
\end{itemize}
\begin{description}[labelindent=1cm,leftmargin=\onelen,labelwidth=1cm]
\myitem{nonterminal}{:} \texttt{textit{formula}}
\end{description}


Where the \ecode{formula} is any syntactic rules for parsing the \ecode{nonterminal}.
Furthermore,  the rest of this manual will reference specific terminals, and nonterminals inline using the same monospace font and formatting.
