\section{Lexical Conventions}

\subsection{White Space}
White space in the DERP language is interpreted in one of three ways,
depending upon its context. Any whitespace character (that is, any
character in the set [\textbackslash r, \textbackslash n, \textbackslash t, ‘ ‘]) can be used as a part of a
string literal used to identify something to the interpreter.
The whitespace character \textbackslash n (a unix-style newline) following any sequence
of non-whitespace text indicates the end of a DERP statement to be parsed.
Any whitespace character in any other context is ignored.


\subsection{Comment Syntax}
DERP does not support in-code comments of any sort.

\subsection{Common Syntax Elements}
A number of common keywords and phrases exist in the DERP language and are
used in multiple expressions. Some of them are just common definitions that
are useful to have defined, and some are optional syntax elements.
The optional syntax elements are injected into expressions to make them
sound more like standard English speech when spoken, but they are often not
required for the syntax to be valid. The common syntax elements and keywords
are defined here.

\begin{itemize}[leftmargin=1in]
    \item[\nonterminal{article}] \bnf{:} \terminal{a} \bnf{|} \terminal{an} \bnf{|} \terminal{the}
    \item[\nonterminal{string}] \bnf{:} \bnf{"[a-zA-Z]+"}
    \item[\nonterminal{digit}] \bnf{:} \bnf{[0-9]}
    \item[\nonterminal{number}] \bnf{:} \nonterminal{digit}\bnf{+([.,]}\nonterminal{digit}\bnf{+)?}   
\end{itemize}
