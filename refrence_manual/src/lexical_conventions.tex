\section{Lexical Conventions}

\subsection{White Space}
White space in the DERP language is interpreted in one of three ways,
depending upon its context. Any whitespace character (that is, any
character in the set [\textbackslash r, \textbackslash n, \textbackslash t, ‘ ‘]) can be used within a
string literal which identifies something to the interpreter.
The whitespace character \textbackslash n (a unix-style newline) following any sequence
of non-whitespace text indicates the end of a DERP statement to be parsed.
Whitespace characters in any other context are ignored.


\subsection{Comment Syntax}
DERP does not support in-code comments.

\subsection{Common Syntax Elements}
A number of common keywords and phrases exist in the DERP language and are
used in multiple syntax rules. Some of them are merely common definitions that
are useful to have globally defined, and some are optional syntax elements.
The optional syntax elements can be injected into statements to make them
sound more like standard English speech though they are not
required for the syntax to be valid. The common syntax elements and keywords
are defined here.

\begin{itemize}[leftmargin=1in]
    \item[\nonterminal{article}] \bnf{:} \terminal{a} \bnf{|} \terminal{an} \bnf{|} \terminal{the}
    \item[\nonterminal{string}] \bnf{:} \bnf{"[a-zA-Z]+"}
    \item[\nonterminal{digit}] \bnf{:} \bnf{[0-9]}
    \item[\nonterminal{number}] \bnf{:} \nonterminal{digit}\bnf{+([.,]}\nonterminal{digit}\bnf{+)?}   
\end{itemize}
