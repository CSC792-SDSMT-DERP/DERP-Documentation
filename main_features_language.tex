\subsection{Main Features}
\subsubsection{Composable}
DERP-QL is composable on several levels. First of all, selections are composed of several \textit{selection statements}.
Selections always start with an \texttt{Add} statement. \texttt{Add} statements always supply at least one source to pull postings from. Additionally, they may restrict the postings to a set of criteria. While selections always start with an \texttt{Add} statement, the selection can be further altered by any number of \texttt{Add} or \texttt{Remove} statements. 

Once a selection has been built, it can be saved under a given name. Any saved selection can then be used as a source for another selection. For example, the above example could be split into multiple selections.

\begin{lstlisting}
Create a new selection
    Add posts from subreddit "nba" or subreddit "basketball" on "Lakers"
    Save as "Lakers news"
Stop
Create a new selection
    Add posts from "Lakers news"
    Remove posts with under 1000 upvotes
    Remove posts on "Lebron James"
    Save as "Lakers news without Lebron"
Stop
\end{lstlisting}

Composing selections makes it easy to assign a name to a number of sources. For example, the first selection in the example immediately above could be broken into two selections.

\begin{lstlisting}
Create a new selection
    Add posts from subreddit "nba" or subreddit "basketball" 
    Save as "basketball subreddits"
Stop
Create a new selection
    Add posts from "basketball subreddits" on "Lakers"
    Save as "Lakers news"
Stop
Create a new selection
    Add posts from "Lakers news"
    Remove posts with under 1000 upvotes
    Remove posts on "Lebron James"
    Save as "Lakers news without Lebron"
Stop
\end{lstlisting}

While reusing selections is handy, it is even more useful to reuse the same filter criteria across multiple sources.
While selections must pull from at least one source with an \texttt{Add} statement, \textit{criteria} are purely logical constructs. \textit{Criteria} provide a means to compose search criteria without pulling posts from any individual source. The selection above can be rewritten in terms of criteria.

\begin{lstlisting}
Create new criteria
    Add posts on "Lakers"
    Remove posts with under 1000 upvotes
    Remove posts on "Lebron James"
    Save as "Lakers without Lebron"
Stop
Create new selection
    Add posts from subbredit "nba" or subreddit "basketball" \ 
        matching "Lakers without Lebron"
    Save as "Lakers news without Lebron"
Stop
\end{lstlisting}
Note: "\textbackslash" is not a part of DERP-QL. However, "\textbackslash" here means the statement continues on the next line.

\subsection{Interactive}
DERP-QL is meant to be used in an interactive manner either through a REPL (Read Evaluate Print Loop) or a HESL (Hear Evaluate Speak Loop). DERP-QL specifies keywords and semantics that a DERP-QL REPL/HESL must support. Of particular note are the \texttt{Read} and \texttt{Recall} keywords. \texttt{Read} signals the interpreter to execute a selection. For example, \texttt{Read "Lakers news without Lebron"} pauses the interpretation of the containing DERP-QL program, and begins the interpreter dependent handling of the stored selection "Lakers news without Lebron". The \texttt{Read} command may also be used while creating a new selection to check the output of the selection being built. 

\texttt{Recall}, on the other hand, makes programming easier for voice users. \texttt{Recall} causes the interpreter to list out the selection statements which make up a given selection. This command is especially helpful in the selection and criteria creation environments when programming by voice. 

\subsection{Extensible}


\begin{enumerate}
\item Syntax is similar to natural language
\item Syntax is compatible with integration with voice control
\item Ability to save selections, criteria, and sources
\item Ability to combine saved selections, criteria, and sources
\item Can value copy saved selections into the selection currently being built
\item Modal interpreter
\item Extensible source design, defaults support Reddit, can create modules for other sources
\item Syntax is tolerant of articles and some 'natural' variants of keywords
\item Capable of creating a selection on a variety of criteria (fields?)
\item tolerant of missing fields on merged selections (i.e. posts are missing tags)
\item ordering of results can be controlled, with source specific defaults
\end{enumerate}

