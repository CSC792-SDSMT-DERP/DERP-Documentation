\subsection{Main Features}
DERP's main features revolve around being composable, interactive, and extensible. 
\subsubsection{Composable}
\label{subsub:Composable}
DERP-QL is composable on several levels. First of all, selections are composed of several \textit{selection statements}.
Selections always start with an \texttt{Add...From} statement. \texttt{Add...From} statements supply at least one source to pull postings from. Additionally, they may restrict the postings to a set of criteria. While selections always start with an \texttt{Add} statement, the selection can be further altered by any number of \texttt{Add} or \texttt{Remove} statements. 

Once a selection has been built, it can be saved under a given name. Any saved selection can then be used as a source for another selection. For example, the above example could be split into multiple selections.

\begin{lstlisting}
Create a new selection
    Add posts from subreddit "nba" or subreddit "basketball" on "Lakers"
    Save as "Lakers news"
Stop
Create a new selection
    Add posts from "Lakers news"
    Remove posts with under 1000 upvotes
    Remove posts on "Lebron James"
    Save as "Lakers news without Lebron"
Stop
\end{lstlisting}

Composing selections makes it easy to assign a name to a number of sources. For example, the first selection in the example immediately above could be broken into two selections.

\begin{lstlisting}
Create a new selection
    Add posts from subreddit "nba" or subreddit "basketball" 
    Save as "basketball subreddits"
Stop
Create a new selection
    Add posts from "basketball subreddits" on "Lakers"
    Save as "Lakers news"
Stop
Create a new selection
    Add posts from "Lakers news"
    Remove posts with under 1000 upvotes
    Remove posts on "Lebron James"
    Save as "Lakers news without Lebron"
Stop
\end{lstlisting}

While reusing selections is handy, it is even more useful to reuse the same filter criteria across multiple sources.
While selections must pull from at least one source with an \texttt{Add...From} statement, \textit{criteria}\label{create:criteria} are purely logical constructs. \textit{Criteria} provide a means to compose search criteria without pulling posts from any individual source. The selection above can be rewritten in terms of criteria.

\begin{lstlisting}
Create new criteria
    Add posts on "Lakers"
    Remove posts with under 1000 upvotes
    Remove posts on "Lebron James"
    Save as "Lakers without Lebron"
Stop
Create new selection
    Add posts from subbredit "nba" or subreddit "basketball" \ 
        matching "Lakers without Lebron"
    Save as "Lakers news without Lebron"
Stop
\end{lstlisting}
Note: ``\textbackslash'' is not a part of DERP-QL. However, ``\textbackslash'' here means the statement continues on the next line.

\subsubsection{Interactive}
DERP-QL is meant to be used in an interactive manner either through a REPL (Read Evaluate Print Loop) or a HESL (Hear Evaluate Speak Loop). DERP-QL specifies keywords and semantics that a DERP-QL REPL/HESL must support. Of particular note are the \texttt{Read} and \texttt{Recall} keywords. \texttt{Read} signals the interpreter to execute a selection. For example, \texttt{Read "Lakers news without Lebron"} pauses the interpretation of the containing DERP-QL program, and begins the interpreter dependent handling of the stored selection ``Lakers news without Lebron''. The \texttt{Read} command may also be used while creating a new selection to check the output of the current selection being built.

\texttt{Recall}, on the other hand, makes programming easier for voice users. \texttt{Recall} causes the interpreter to list out the statements which make up a given selection. This command is especially helpful in the selection and criteria creation environments when programming by voice. 

In order to promote human interaction, several words are parsed as filler including lone articles \texttt{a, an, the} and the word \texttt{with}.

\subsubsection{Extensible}\label{sssec:Extensible}
Source modules may define their own grammars which are concatenated into the overall language. For example, the reddit source module defines the \texttt{subreddit} keyword. The subreddit keyword takes the next string and uses it to select a portion of reddit to use as a source.

% TODO: Expand
