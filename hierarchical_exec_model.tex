\subsection{Hierarchical Data}
Data structure flow model: $$website \rightarrow boards (subreddits) \rightarrow posts$$\\
DERP is a declarative language, so the main focus of a programmer will be the logic of the statements to perform queries, otherwise known as selections. This means that the task of organizing queries and their results is performed by the interpreter executing the REPL. The REPL will accept statements to define sources, identify defined sources, and request information from said sources. The interpreter itself will verify that each statement is valid and conforms to the standards of the current mode before executing it, taking into account the set of modules loaded for news sources.

Upon execution of a statement, the interpreter may register new sources or perform a search of the loaded news sources with the criteria specified. When executing a query, it will retrieve only enough information for the the user to identify an article's defining information. The defining information of articles will be tags such as its title, date published, and source. Statements following a query may perform summaries of the query results, start new queries, give details about each article, or otherwise perform some operation on the retrieved data.


\subsubsection{Modes}
The REPL will have three modes of evaluation; the mode used to interpret a given statement will be dependent upon the language keywords provided in the statement. These modes are the main mode, selection mode, and criteria mode.

\paragraph{Main Mode}

The DERP interpreter will start in the main mode, also known as the top-level mode. This mode allows users to clear and read statements, as well as enter other modes. From this state, it is valid to switch the interpreter to either the selection more or the criteria creation mode. In this mode, language keywords which end a program should indicate that the REPL is to exit.

\paragraph{Selection Mode}

The selection mode will be used to add modules, add and remove information from the current query, and save or read the query so far. In this mode, language keywords which end a program should indicate that the REPL is to return to the top-level mode. Examples of using this mode are found in \autoref{subsub:Composable}. 

\paragraph{Criteria Mode}

The criteria mode allows will be used to create logical statements that use the 'add', 'remove', and 'save' keywords. In this mode, language keywords which end a program should indicate that the REPL is to return to the top-level mode. Details can be found in \autoref{subsub:Composable}.


\subsubsection{Modules}

The interpreter will also allow for language modules to be loaded during run-time. These modules will add language keywords which allow users to use new kinds of news sources or query criteria. Examples of modules can be found in \autoref{sssec:Extensible}

\paragraph{News Module }

The news source is going to a be a modular component of the interpretor that can be defined by users. This will be a module in python that will define how the interpretor will interface with the source, what are the keywords associated with it, filters and search criteria will also be defined. For example reddit has a Python api that can be interfaced with and will have kewords that get defined such as subreddit and posts. It will also define what is considered a criteria, what a filter can exclude and so on. For simplicity, Reddit will be implemented initally and others added if time permits. When this becomes a HESL, the modules will be built into the interpreter and periodic updates will happen to allow for more sources of news to be suppported.\\


\paragraph{Knowledge Module}
The knowledge modules are used when the user wants to search through articles that were previously fetched using selection statements with news modules. This module would define how articles a are searched through and filtered locally. It would get the contents of the article at this point and filter tham further by the searchable criteria that the user stated.
